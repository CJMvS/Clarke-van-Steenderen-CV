\documentclass{article}
\usepackage{scimisc-cv}

%% These are custom commands defined in scimisc-cv.sty
\cvname{\Large{Clarke JM van Steenderen, BSc, BSC (Hons), MSc}}
\cvpersonalinfo{
Department of Zoology and Entomology, Rhodes University, Grahamstown, 6139, South Africa 
}

\begin{document}

\makecvtitle 

\section{Personal Information}
\begin{description}[widest=Driver's licence:]
\item[Email \Letter:] vsteenderen@gmail.com
\item[Cell \Telefon:] +27 (0)72 529 0732
\item[GitHub \faGithub:] \url{https://github.com/CJMvS}
\item[Twitter \faTwitter:] @ClarkeJMVS
\item[ORCID ID:] \url{https://orcid.org/0000-0002-4219-446X} 
\item[Citizenship:] South African
\item[Driver’s licence:] Code B
\end{description}

\section{Research Interests}

\begin{itemize}
\item Entomology
\item Biological control of invasive weeds and/or insect pests
\item Answering evolutionary questions using phylogenetics and molecular tools
\item Creating user-friendly applications in R and Python to streamline data processing

\end{itemize}

\section{Skills}

\begin{itemize}
\item \textbf{Molecular biology:} DNA extraction, PCR, phylogenetics, genetic barcoding, fragment analyses (ISSR, SSR) and data processing
\item \textbf{Computational:} Programming in Python and R, R Shiny Apps
\end{itemize}
 
\section{Courses Completed}

\begin{itemize}

\item Advanced Statistics in R, accredited through Rhodes University (August 2020)
\item Invasive Weeds Short Course, accredited through Rhodes University (October 2018)
\item Phylogenetics beginner and advanced workshop, Stellenbosch University (June 2018)

\end{itemize}

\section{Education}

\begin{itemize}
\item PhD, Entomology, Rhodes University, 2020-present 
\item MSc, Entomology, Rhodes University, 2018-2019
\item BSc Hons (with distinction, top honours student), Entomology, Rhodes University, 2017  
\item BSc, Zoology and Entomology (with a distinction in Entomology), Rhodes University (EC Province), 2014-2016
\item Selly Park Secondary School, Rustenburg (NW Province), 2008-2012, seven distinctions in matric (English first language, Afrikaans second additional language, Mathematics, Life Science, Physical Science, History, Life Orientation)
\end{itemize}
 
\section{Teaching, Work, and Project Experience}
\begin{itemize}

\item \textbf{Postgraduate projects}
\begin{itemize}
    \item \textbf{PhD} A genetic investigation of the native stem-boring \textit{Tetramesa} wasps in South Africa: identifying potential biological control agents of invasive grasses
    \item \textbf{MSc} Using genetic barcoding methods to identify the different species and intra-specific lineages of \textit{Dactylopius} Costa used as biological control agents of invasive Cactaceae
\end{itemize}
\item \textbf{Phylogenetics tutorials}
I delivered a series of 1-1.5 hour/week tutorials to colleagues within the Centre for Biological Control (CBC) research group, September - October 2020. See my \href{https://github.com/CJMvS/CBC_Tutorials}{GitHub repository}.
\item \textbf{Undergraduate and Honours projects}
\begin{itemize}
    \item The efficacy of a host-specific granulovirus (PlxyGV) under field conditions on the biological control of the diamondback moth (\textit{Plutella xylostella} L.)
    \item The extent of microplastic pollution along the South African coastline using indigenous (\textit{Perna perna}) and invasive (\textit{Mytilus galloprovincialis}) mussel tissue as indicators
    \item The biological control of water hyacinth under eutrophic conditions and the effects of different herbivorous feeding guilds on plant health and defenses
    \item The determination of the efficacy of a unisexual vs a bisexual sterile insect release for the false codling moth (\textit{Thaumatotibia leucotreta} Meyrick) (Lepidoptera: Tortricidae); a major citrus pest in South Africa. I also compared mating preference and successful mating ability between sterile and wild adults
\end{itemize}
\item \textbf{Demonstrating} 
Cell biology 101, Zoology 102, Zoology 201 and 202, Entomology 201, 202, 301 and 302, Honours R statistics courses; Department of Zoology and Entomology, Rhodes University (2017, 2018, 2019, 2020, 2021).
Duties included assisting undergraduate students with their practical classes and fieldwork, and marking their scripts and assignments
\item \textbf{Postgraduate student representative (MSc students)}
Centre for Biological Control, Department of Zoology and Entomology, Rhodes University (2018-2019)
\item \textbf{Library assistant}
Student assistant at the Rhodes University main library loans desk (2018)
\item \textbf{Committee member} Environmental Representative (House Committee) for my residence (2016) at Rhodes University (Kimberley Hall)
\item \textbf{Volunteering} Week-long plant collecting excursion in the Stormberg Mountains for the Botany Department, Rhodes University (2014)
\item \textbf{Tutoring} Computer skills on FirstTutors South Africa (2021-);
Mathematics, physical science and life science to grade 10 and 11 homeschooling students (Rustenburg, North West Province, 2013)  
\end{itemize}

\section{Awards}
\begin{itemize}

\item Departmental PhD bursary through the Centre for Biological Control (CBC) (2020 – 2022)
\item Awarded an NRF bursary for an MSc degree (2018 - 2019)
\item Academic Colours (Rhodes University, 2017)
\item Entomological Society of Southern Africa Student Award: best Honours student in Entomology, Rhodes University, 2017
\item Ewer prize for Zoology, Department of Zoology and Entomology, Rhodes University, 2017
\item Academic Excellence Award, Postgraduate Hall, 2017
\item Recipient of the Ada and Bertie Levenstein bursary for 2017, 2018, and 2019
\item Awarded the Henderson bursary for 2017 and 2018
\item Placed on the Dean’s list of the Faculty of Science for Academic Merit for the years 2014, 2015, 2016, and 2017, Rhodes University
\item Academic Excellence Award (Kimberley Hall, Rhodes University, 2016)
\item Distinguished Excellence Award (Kimberley Hall, Rhodes University, 2016)
\item The Most Inspiring Person Award (Kimberley Hall, Rhodes University, 2015)
\item Top Student’s Award (Kimberley Hall, Rhodes University, 2015)
\item Academic Excellence Award (Kimberley Hall, Rhodes University, 2015)
\item Academic Half Colours (Rhodes University, 2015)
\item ``Top 5" student award received from grade 8 through to matric.
\item Represented the North West Province in the South African National Chess Championships in 2010 (UCT) and 2011 (UJ)

\end{itemize}

\section{Conference and other Presentations}

\begin{itemize}

\item Sutton, G.F., \textbf{van Steenderen, C.J.M}., Canavan, K., Yell, L., and Paterson, I.D. South Africa is a hotspot for previously unknown stem-boring wasps of grasses (\textit{Tetramesa}; Eurytomidae). Grassland Society of Southern Africa, 56th Annual Congress. July 26 - 30 2021. 
\item \textbf{van Steenderen, C.J.M.}, Paterson, I.D., Sutton, G.F., and Canavan, K. A genetic investigation of the native stem-galling \textit{Tetramesa} Walker (Hymenoptera: Eurytomidae) in South Africa, and their potential use as biological control agents. 22nd Hybrid Congress of the Entomological Society of Southern Africa (ESSA). 28 June - 1 July 2021.
\item \textbf{van Steenderen, C.J.M.}, Paterson, I.D. and Edwards, S. Cochineal identification: how molecular techniques can distinguish between biological control agents and agricultural pests. Second International Congress of Biological Control (ICBC2), Davos, Switzerland [virtual conference]. Biological control of cactus pests and pest cacti online session. 26 - 30 April 2021.
\item \textbf{van Steenderen, C.J.M.}, Moore, S.D., Marsberg, T., Peyper, M., Kirkman, W., and Hill, M.P. Are sterile females in an SIT programme for FCM beneficial to its success? Will be presented by Sean Moore at the 11th Citrus Research Symposium, Drakensberg, August 2021.
\item \textbf{van Steenderen, C.J.M.}, Paterson, I.D. and Edwards, S. The genetic barcoding of the species and lineages of \textit{Dactylopius} Costa (Hemiptera: Dactylopiidae). The National Symposium on Biological Invasions (15-17 May 2019), Tulbagh, Western Cape. 
\item Cactus Working Group (CWG), Botanical Gardens, Pretoria, 14 November 2018
\item Guest talk at Victoria Girls High School, 29 October 2018, Grahamstown
\end{itemize}

\section{Peer-Reviewed Publications}

\begin{itemize}

    % \item Owen, C., Sutton, G.F., Martin, G.D., \textbf{van Steenderen, C.J.M.}, Coetzee, J.A. Sample size planning for insect critical thermal limit studies. 2021. \textit{Journal}. doi:\url{}.

    \item \textbf{van Steenderen, C.J.M.}, Paterson, I.P., Edwards, S., and Day, M.D. Addressing the red flags in cochineal identification: the use of molecular techniques to identify cochineal insects that are used as biological control agents for invasive alien cacti. 2021. \textit{Biological Control}. doi: \url{https://doi.org/10.1016/j.biocontrol.2020.104426}
\end{itemize}

\section{R Packages and Shiny Applications}
    \begin{itemize}
    \item \textbf{van Steenderen, C.J.M.} BinMat:  Processes Binary Data Obtained from Fragment Analysis Methods. \href{https://cran.r-project.org/web/packages/BinMat/}{CRAN repository}, 2020 (also see the \href{https://clarkevansteenderen.shinyapps.io/BINMAT/?_ga=2.18474297.1423947265.1599643639-2015717805.1599643639}{Shiny Application})
    \item \href{https://clarkevansteenderen.shinyapps.io/Dactylopius_ID_version_1/?_ga=2.47818023.1423947265.1599643639-2015717805.1599643639}{Dacty-ID} Identifies a query genetic sequence (12S, 18S, or COI) for cochineal species, relative to the genetic database created in my MSc project
    \item \href{https://clarkevansteenderen.shinyapps.io/BarcodeTester/?_ga=2.47818023.1423947265.1599643639-2015717805.1599643639}{Genetic Barcode Tester} Tests the accuracy of genetic barcode data
    \item \href{https://clarkevansteenderen.shinyapps.io/ThermalSampleR_Shiny/}{ThermalSampleR} Extrapolates thermal tolerance data from entomological studies to estimate optimal sample sizes  
    \item \href{https://github.com/CJMvS/spede-sampler}{SPEDE-SAMPLER GMYC} Runs the GMYC species delimitation method on multiple resampled tree files
    \end{itemize}
    
\section{Reviewer}
\href{https://www.journals.elsevier.com/biological-control}{Biological Control} (1 paper) [Impact factor: 2.754] \\
\href{https://onlinelibrary.wiley.com/journal/13653180}{Weed Research} (1 paper) [Impact factor: 2.011]

\section{Popular Articles}
\begin{itemize}
    \item \textbf{van Steenderen, C.J.M.} Joe Dispenza’s \textit{Becoming Supernatural}:
How Common People Are Being Misled. \textit{Skeptical Inquirer} Vol. 40, No. 4. July/August 2020. \url{https://skepticalinquirer.org/authors/clarke-van-steenderen/}
\end{itemize}

\section{Other Skills}
\begin{description}[widest=Languages]
\item[Software] Microsoft Word, Excel, and PowerPoint, LaTex, GitHub, and a variety of phylogenetic software programs
\item[Languages] English: professional proficiency.  Afrikaans: conversational. 
\end{description}

\section{Memberships}
\begin{itemize}
    \item International Organization for Biological Control-Afrotropical Regional Section (IOBC-ATRS) for 2020 (Membership number ATRS-034)
    \item Golden Key Society
\end{itemize}

\section{Hobbies}
Playing chess, squash, tennis, \href{https://www.youtube.com/channel/UC2qRVnmvs5yVTdaBes1ndOA}{playing guitar in a duo band}, reading, writing, drawing, jogging, spending time with my mates.

% \section{References}
% \begin{itemize}
%     \item Prof. Martin Hill, Head of the Entomology Department and Centre for Biological Control, Rhodes University. Honours supervisor (2017) | m.hill@ru.ac.za | +27 (0)46 603 8712
%     \item Prof. William Froneman, HoD of the Department of Zoology and Entomology | w.froneman@ru.ac.za | +27(0) 46 603 8959
%     \item Prof. Julie Coetzee, Lecturer/researcher, Department of Zoology and Entomology, Centre for Biological Control, Rhodes University, Honours supervisor (2017), PhD co-supervisor (2020-present) | julie.coetzee@ru.ac.za 
%     \item Prof. Iain Paterson, Centre for Biological Control researcher, Rhodes University, PhD and MSc main supervisor (2018-2019 and 2020-present), \\ i.paterson@ru.ac.za | +27(0) 46 603 8098
%     \item Dr. Grant Martin, Centre for Biological Control researcher, Rhodes University, \\
%     g.martin@ru.ac.za | +27 (0)46 603 8702 
%     \item Dr. Shelley Edwards, Zoology and Entomology Molecular Lab (ZEML), Rhodes University, MSc co-supervisor (2018-2019), \\ s.edwards@ru.ac.za | +27(0) 46 603 8086
%     \item Prof. Martin Villet, Department of Zoology and Entomology, m.villet@ru.ac.za | +27 (0)46 603 8527
%     \item Ms. Erica da Silva, Senior demonstrator for Zoology 101 (2017), Rhodes University, erica.dasilva526@gmail.com | +27(0) 72 625 2532

% \end{itemize}

\end{document}
